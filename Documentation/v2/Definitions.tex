% german names
\usepackage{ngerman}
% utf-8
\usepackage{polyglossia}
\setmainlanguage[spelling=new]{german}
\usepackage{fontspec}
\usepackage{pdfpages}

% In order to remove scr warning
\usepackage{scrhack}

%Needed to wrap text around images, used for personas
\usepackage{wrapfig}

\usepackage{enumitem}

%Used to avoid page breaks with each chapter
\usepackage{etoolbox}
\makeatletter
\patchcmd{\chapter}{\if@openright\cleardoublepage\else\clearpage\fi}{}{}{}
\makeatother

% colors
\usepackage{color}
\definecolor{grey}{rgb}{0.2,0.2,0.2}
\definecolor{orange}{rgb}{1,0.3,0}
\definecolor{turqoise}{rgb}{0,0.7,0.5}
\definecolor{OliveGreen}{rgb}{0.2,0.7,0.4}
\definecolor{Plum}{rgb}{0.52,0,0.7}

% Correctly break urls at hyphens
\usepackage[hyphens]{url}

% Special Link color (same as cite color)
\usepackage[colorlinks=true, citecolor=magenta, linkcolor=magenta, urlcolor=magenta] {hyperref}

%Black to save some costs
%\usepackage[colorlinks=true, citecolor=black, linkcolor=black, urlcolor=black] {hyperref}


% Use more than one optional parameter in a new commands
\usepackage{xargs}

% Adds Todos to output
\usepackage[colorinlistoftodos,prependcaption,textsize=tiny]{todonotes}
\newcommandx{\unsure}[2][1=]{\todo[linecolor=red,backgroundcolor=red!25,bordercolor=red,#1]{#2}}
\newcommandx{\change}[2][1=]{\todo[linecolor=blue,backgroundcolor=blue!25,bordercolor=blue,#1]{#2}}
\newcommandx{\info}[2][1=]{\todo[linecolor=OliveGreen,backgroundcolor=OliveGreen!25,bordercolor=OliveGreen,#1]{#2}}
\newcommandx{\improvement}[2][1=]{\todo[linecolor=Plum,backgroundcolor=Plum!25,bordercolor=Plum,#1]{#2}}
\newcommandx{\thiswillnotshow}[2][1=]{\todo[disable,#1]{#2}}

% To display figures exactly there where you want them to be
\usepackage[section]{placeins}

%Place subsections figures where you want them to be
\makeatletter
\AtBeginDocument{%
  \expandafter\renewcommand\expandafter\subsection\expandafter{%
    \expandafter\@fb@secFB\subsection
  }%
}
\makeatother

% code listings
\usepackage{listings}
\newcommand*\lstinputpath[1]{\lstset{inputpath=#1}}
\lstinputpath{Code}

\renewcommand{\lstlistingname}{Code}
\renewcommand{\lstlistlistingname}{Quellcodeverzeichnis}

\renewcommand*{\figureautorefname}{Ab\-bil\-dung}
\renewcommand*{\sectionautorefname}{Ab\-schnitt}
\renewcommand*{\chapterautorefname}{Ka\-pi\-tel}
\renewcommand*{\subsectionautorefname}{Ab\-schnitt}
\renewcommand*{\subsubsectionautorefname}{Ab\-schnitt}
\newcommand{\lstnumberautorefname}{Zeile}


\lstdefinelanguage{scala}{
  morekeywords={abstract,case,catch,class,def,%
    do,else,extends,false,final,finally,%
    for,if,implicit,import,match,mixin,%
    new,null,object,override,package,%
    private,protected,requires,return,sealed,%
    super,this,throw,trait,true,try,%
    type,val,var,while,with,yield},
  otherkeywords={=>,<-,<\%,<:,>:,\#,@},
  sensitive=true,
  morecomment=[l]{//},
  morecomment=[n]{/*}{*/},
  morestring=[b]",
  morestring=[b]',
  morestring=[b]"""
}

\lstdefinelanguage{buildSbt}{
  morekeywords={libraryDependencies, name, class},
  otherkeywords={+=, \%, ACCEPT, DENY},
  sensitive=true,
  morecomment=[l]{//},
  morecomment=[n]{/*}{*/},
  morestring=[b]",
  morestring=[b]'
}

% Style languages
\lstdefinestyle{Code}{
	language=sql,
	basicstyle={\ttfamily \small},
	breaklines=true,
	commentstyle=\color{grey},
	keywordstyle=\color{orange},
	numbers=left,
	showspaces=false,
	captionpos=t,
	abovecaptionskip=12pt,
	stringstyle=\color{turqoise},
	xleftmargin=20pt
}
\lstdefinestyle{JSON}{
	string=[s]{"}{"},
	stringstyle=\color{blue},
	comment=[l]{:},
	commentstyle=\color{black},
	showspaces=false,
	captionpos=t,
	abovecaptionskip=12pt,
	xleftmargin=20pt
}
\lstdefinestyle{scala}{%
	basicstyle={\ttfamily \small},
	breaklines=true,
	commentstyle=\color{grey},
	keywordstyle=\color{orange},
	numbers=left,
	showspaces=false,
	stringstyle=\color{turqoise}
}

\lstdefinestyle{buildSbt}{%
	basicstyle={\ttfamily \small},
	breaklines=true,
	commentstyle=\color{grey},
	keywordstyle=\color{orange},
	showspaces=false,
  stringstyle=\color{turqoise}
}

\lstset{%
	language=bash,
	basicstyle={\ttfamily \small},
	breaklines=true,
	postbreak=\mbox{\textcolor{red}{$\hookrightarrow$}\space},
	commentstyle=\color{grey},
	keywordstyle=\color{black},
  stringstyle=\color{black},
	otherkeywords={},
	numbers=left,
	showspaces=false,
	xleftmargin=\fboxsep,
  xrightmargin=-\fboxsep,
  aboveskip=15pt,
  frame=b,
  escapeinside={~}{~}
}

% This is necessary in order to avoid stupid latex errors, which are caused by importing gensymb package.
\usepackage{textcomp}

% Used for the degree symbol
\usepackage{gensymb}

% Place caption beneath listing
\usepackage{caption}

% Adding the source of a picture beneath it
\usepackage{subcaption}

%for images that are wrapped with text
\usepackage{wrapfig}
\usepackage{ragged2e}
\DeclareCaptionFormat{myformat}{#1#2\\#3}
\DeclareCaptionFont{white}{\color{white}}
\DeclareCaptionFormat{listing}{%
  \parbox{\textwidth}{\colorbox{darkgray}{\parbox{\textwidth}{#1#2#3}}}}
\captionsetup[lstlisting]{format=listing,labelfont=white,textfont=white, skip=8pt}
\captionsetup[figure]{skip=0pt}

% For adding the titelpage as a pdf
\usepackage{pdfpages}

% graphics
\usepackage{graphicx}
\graphicspath{{Images/}}

\usepackage{float}

% BibTex lib - used for citation
%\usepackage{cite}

% Alternative to using common cite.

% Continuous footnote numbering (usually reset after chapter)
\usepackage{chngcntr}
\counterwithout{footnote}{chapter}

\usepackage[automark,headsepline,autooneside=false]{scrlayer-scrpage}
\clearpairofpagestyles{}
\ihead{\leftmark}
\chead{}
\ohead{\ifstr{\rightmark}{\leftmark}{}{\rightmark}}
%\ifoot{Bachelorthesis --- Tobias Kerst}
%\cfoot{}
\ofoot*{\pagemark}
\pagestyle{scrheadings}
\setkomafont{pageheadfoot}{\normalfont}

% Better layout size at footer and header
% footskip = is the dimension used as the baselineskip in the footer
\setlength\footskip{2cm}
\setlength\textheight{21.5cm}
%distance between the bottom of the text block and the top of the footer is
\setlength{\skip\footins}{1cm}
\setlength{\headsep}{1.5cm}
% Creates a bigger margin from page to header
%\usepackage[top=4cm]{geometry}
\setlength{\voffset}{0.8cm}

% Better footnote layout
\usepackage[bottom,hang]{footmisc}
\setlength{\footnotemargin}{1.2em}

\usepackage{lipsum}

\setkomafont{disposition}{\normalfont\bfseries}

% for verbatiminput
\usepackage{verbatim}

% Used for cross-referencing between different files
\usepackage{xr}

% German quotation marks
\let\oldquote'
\newif\ifquoteopen
\catcode`\'=\active
\makeatletter
% we have to redefine \pr@m@s to use an active '
\def\pr@m@s{%
  \ifx'\@let@token
    \expandafter\pr@@@s
  \else
    \ifx^\@let@token
      \expandafter\expandafter\expandafter\pr@@@t
    \else
      \egroup
    \fi
  \fi}
\protected\def'{%
  \ifmmode
    \expandafter\active@math@prime
  \else
    \expandafter\active@text@prime
  \fi}
\def\active@text@prime{%
   \@ifnextchar'{%
     \ifquoteopen
       \global\quoteopenfalse\grqq\expandafter\@gobble
     \else
       \global\quoteopentrue\glqq\expandafter\@gobble
     \fi
   }{%
     \ifquoteopen
       \global\quoteopenfalse\grq\xspace
     \else
       \global\quoteopentrue\glq
     \fi
   }%
}
\makeatother

% not yet used
%\input{src/cmd}

% Defines a macro \Autoref to allow multiple references to be passed to \autoref
\makeatletter
\newcommand\Autoref[1]{\@first@ref#1,@}
\def\@throw@dot#1.#2@{#1}% discard everything after the dot
\def\@set@refname#1{%    % set \@refname to autoefname+s using \getrefbykeydefault
    \edef\@tmp{\getrefbykeydefault{#1}{anchor}{}}%
    \def\@refname{\@nameuse{\expandafter\@throw@dot\@tmp.@autorefname}n}%
}
\def\@first@ref#1,#2{%
  \ifx#2@\autoref{#1}\let\@nextref\@gobble% only one ref, revert to normal \autoref
  \else%
    \@set@refname{#1}%  set \@refname to autoref name
    \@refname~\ref{#1}% add autoefname and first reference
    \let\@nextref\@next@ref% push processing to \@next@ref
  \fi%
  \@nextref#2%
}
\def\@next@ref#1,#2{%
   \ifx#2@ und~\ref{#1}\let\@nextref\@gobble% at end: print and+\ref and stop
   \else, \ref{#1}% print  ,+\ref and continue
   \fi%
   \@nextref#2%
}
\makeatother

% avoids schusterjungen and hurenkinder
\usepackage[all]{nowidow}
\clubpenalty10000
\widowpenalty10000
\displaywidowpenalty=10000
