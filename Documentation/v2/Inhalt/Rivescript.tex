\chapter{Rivescript} \label{sec:rivescript}
Rivescript beschreibt sich selbst als einfache Skriptsprache, die Chatbots Intelligenz gibt.~\footnote{\url{https://www.rivescript.com/about}} Es ermöglicht in einem gewissen Rahmen eine Konversation des Benutzers mit dem Chatbot innerhalb eines Kontextes aufrecht zu erhalten. Eine weitere Funktion ist zum Beispiel das Merken von Variablen.

Bei dieser Projektarbeit werden solche erweiterten Funktionen nicht benötigt. Das Hauptszenario ist, dass der Benutzer dem Bot eine Frage stellt. Diese Frage wertet der Bot aus und liefert die entsprechenden Informationen zurück. Die Unterhaltung beschränkt sich also meistens auf eine Frage und eine Antwort.

\section{Vorstellung}
Rivescript wird einfach mithilfe eines Texteditors in eine \texttt{.rive}-Datei geschrieben, die von einem Interpreter ausgelesen und ausgewertet wird. Es werden im Bezug auf den Interpreter derzeit die Sprachen \texttt{Go}, \texttt{Java}, \texttt{JavaScript}, \texttt{Perl} und \texttt{Python} direkt von den Rivescript Entwicklern und \texttt{C\#} und \texttt{PHP} von Drittentwicklern unterstützt. Die erste unterstützte Sprache war Perl.

Rivescript funktioniert nach einem einfachen Trigger-Response-Prinzip. Die Eingabe des Benutzers wird ausgelesen und es wird in festgelegten Ausdrücken gesucht, ob sie mit einem Ausdruck übereinstimmt. Ist dies der Fall, antwortet der Bot mit der entsprechenden Antwort.
Ein einfaches Rivescript-Programm könnte also lauten:
\lstinputlisting[firstline=1, lastline=4, numbers=none, caption=Einfaches Rivescript-Programm, label=basicRive]{riveExamples.rive}

Das ''\textbf{+}''-Symbol stellt dabei die Eingabe des Users, das ''\textbf{-}''-Symbol die Ausgabe des Bots da. So können auf eine Eingabe verschiedene Ausgaben passen, bei denen dann zufällig ausgewählt wird, welche der Bot zurückgibt. Den Ausgaben kann aber auch eine gewisse Gewichtung gegeben werden, die dann eine feste Wahrscheinlichkeit der Antworten vorgibt.
\lstinputlisting[firstline=3, lastline=5, numbers=none, caption=Gewichtete Ausgaben in Rivescript, label=weight]{riveExamples.rive}

Rivescript bietet die Möglichkeit sprachliche Alternativen bekannt zu machen, sodass bei unterschiedlichen Schreibweisen trotzdem eine passende Eingabe gefunden werden kann.
Mit dem Schlüsselwort \emph{! sub} können so Abkürzungen bekannt gemacht werden.
\lstinputlisting[firstline=7, lastline=9, numbers=none, caption=Automatischer Austausch von sinngleichen Wörtern in Rivescript, label=subs]{riveExamples.rive}

Man möchte jedoch nicht nur auf komplett festgesetzte Sätze matchen, sondern auch auf ''vagere'' Aussagen. Dafür stellt Rivescript Wildcards bereit.
\lstinputlisting[firstline=11, lastline=12, numbers=none, caption=Wildcards in Rivescript, label=wildcards]{riveExamples.rive}

Das ''\textbf{*}''-Symbol steht für eine beliebige Anzahl von Wörtern und Zahlen. Dieses kann in der Antwort berücksichtigt werden und damit zum Beispiel, wie hier gezeigt,  als Aufrufparameter einer Methode benutzt werden.
Weitere Wildcards sind das ''\textbf{\_}''- und ''\textbf{\#}''-Symbol. Das ''\textbf{\_}''-Symbol fordert ein Wort ohne Zahlen und Leerzeichen, das ''\textbf{\#}''-Symbol eine Zahl.
Weiterhin gibt Rivescript die Möglichkeiten Arrays anzulegen, die in der Eingabe des Benutzers enthalten sein können, wie zum Beispiel Farben.
\lstinputlisting[firstline=15, lastline=18, numbers=none, caption=Arrays in Rivescript, label=array]{riveExamples.rive}

In Rivescript erhalten die Klammertypen verschiedene Aufgaben. Bei runden Klammern ist es möglich für die Ausgabe des Bots den Inhalt der Klammern als Parameter zu benutzen oder es wie im zuvor genannten Beispiel direkt auszugeben.

Durch die Worte in eckigen Klammern erhält der Eingabesatz eine bessere Struktur und die Wahrscheinlichkeit eine Übereinstimmung zu finden steigt, aber sie sind nicht als mögliche \texttt{<star>}-Parameter nutzbar. Sie sind optional.

Um mehrere Formulierungen in einem Satz abzudecken, ist es möglich Alternativen durch das Oder-Symbol aneinander zu hängen.
\lstinputlisting[firstline=21, lastline=22, numbers=none, caption=Mehrere Formulierungen durch Alternativen in einem Ausdruck, label=alternatives]{riveExamples.rive}

So sind die Formulierungen ''Wie sieht mein Stundenplan aus?'' und ''Was ist mein Stundenplan?'' beide in einem Fall abgedeckt. Hier gilt ebenfalls, dass der Inhalt von runden Klammern als Match verfügbar gemacht werden kann, eckige Klammern wären wieder nur optional. \\
Zu erwähnen gibt es noch, dass Rivescript-Satzzeichen in den erwarteten Eingaben des Users automatisch wegoptimiert. Es ist also lediglich notwendig grammatikalisch korrekte Sätze für die Ausgabe des Bots zu schreiben.

\section{Einbindung und Implementierung}
Die Einbindung in das bestehende Projekt war sehr einfach, da es für Rivescript bereits eine Java-Bibliothek gibt, die dem Projekt hinzugefügt werden kann. Die \texttt{build.sbt}-Datei muss hierbei um die folgende Zeile erweitert werden:
\lstinputlisting[language=buildSbt, style=buildSbt, numbers=none, caption=Auszug der build.sbt, label=buildSbt]{build.sbt}

\subsection{Reagieren auf Freitext}
Da der Bot bis zu diesem Zeitpunkt lediglich auf Commands reagiert hat, was durch die \texttt{onCommand}-Methoden umgesetzt wurde, musste der Bot zusätzlich auf normale Texteingaben reagieren, also auf Textnachrichten, die nicht mit einem Slash beginnen. Das \texttt{Telegrambot4s}-Framework bietet hierfür keine spezielle Methode, aber man kann auf allgemeinen Input reagieren.
Es wurde hierfür also eine neue Klasse erstellt, die zu den Commands gehört und auf Chat-Nachrichten reagiert. 
\lstinputlisting[language=scala, style=scala, caption=Abfangen der Freitextnachrichten in Chat.scala, label=chatScala1, firstline=1, lastline=15]{Chat.scala}

Wie man in \Autoref{line:inputSlash} erkennt, wird auf jede beliebige Nachricht reagiert und geprüft, ob die Nachricht mit einem Slash beginnt. Beginnt die Eingabe mit einem Slash, soll nicht weiter gearbeitet werden, da es hierfür optimierte Klassen und Methoden gibt. Beginnt die Eingabe jedoch nicht mit einem Slash, so handelt es sich also um Freitext. An dieser Stelle wird nun Rivescript eingesetzt um die Eingabe zu interpretieren.

\subsection{Rivescript-Chatbot}
Rivescript stellt eine eigene Klasse zur Verfügung, die genutzt werden kann um den Chatbot zu konfigurieren. Hierbei kann nicht nur eingestellt werden, welche \texttt{.rive}-Dateien der Bot nutzen soll, sondern es können auch Dinge wie \texttt{utf-8}-Support eingestellt werden. Für den korrekten Umgang mit Umlauten in der deutschen Sprache ist dies eine Voraussetzung für den IWINewsBot.

Die einfachste Möglichkeit zum Setzen der Attribute wären Aufrufen von \texttt{Setter}-Methoden auf der von Rivescript bereitgestellten Rivescript-Klasse, um den Code aber modularer und verständlicher  zu machen, wurde eine eigene Klasse erstellt, die von Rivescript erbt und intern alle nötigen Einstellungen trifft. \newpage
\lstinputlisting[language=scala, style=scala, caption=Initialisierung von Rivescript und seiner Routinen, label=initRive]{ChatBot.scala}

Neben dem Setzen von \texttt{utf8} ist vor allem das Bekanntmachen der \texttt{.rive}-Dateien ein Bestandteil dieser Klasse und außerdem das Setzen von Routinen. Diese Routinen werden aus den \texttt{.rive}-Dateien aufgerufen und werden benötigt um programmatische Aufrufe zu tätigen, also Funktionen umzusetzen, die mit Rivescript alleine nicht möglich sind. Dies wird im Kapitel Implementierte Funktionen genauer betrachtet. \\
Der voll konfigurierte Chatbot kann nun genutzt werden, um Freitextnachrichten zu lesen, indem er als Singleton eingebunden wird und die Eingabe bekommt. Als Antwort erhält man entsprechend einen String. Wie dies konkret implementiert ist, wird im folgenden gezeigt:
\lstinputlisting[language=scala, style=scala, caption=Abfangen der Freitextnachrichten in Chat.scala, label=chatScala2, firstline=17, lastline=37]{Chat.scala}

Da der IWINewsBot als \texttt{.jar} gebaut wird und die Rivescript Dateien ein Teil des Programms sind, müssen diese auch in das \texttt{.jar} gepackt werden. Dies geschieht am einfachsten, indem die \texttt{.rive}-Dateien im \texttt{resources/}-Ordner abgelegt werden. \\
Das Laden hat sich als nicht trivial herausgestellt, es war nicht möglich die Dateien einfach als \texttt{File}-Objekte zu lesen, lediglich das Lesen als InputStream war möglich, was jedoch durch Rivescript möglich ist.