\chapter{Fazit}
Im Laufe dieser Projektarbeit konnten ausgiebig Erfahrungen für den Aufbau eines Chatbots mit Rivescript gesammelt werden. Die Vor- und Nachteile werden im Folgenden beschrieben. Außerdem wird aus dem Proof-of-Concept der Sprachnachrichten ein Resümmee gezogen. Im letzten Abschnitt wird ein mögliches weiteres Vorgehen beleuchtet.

\section{Bewertung}
%einfaches Erlernen, schnell
%mehrere Dateien definierbar
%Arrays und Subs
%Kontext
%Routinen → Erweiterung durch eigenen Code
%viele unterstützt Sprachen
%einfache Einbindung

%muss viel im Code passieren, weil Rivescript so basic ist
%beispiel: mehr-wörtrige Parameter
%keine eigenes Logging für Fehlerfälle
%zu viel Freiheit: besser für jeden Parameter eigene Routine oder Parameter in Routine auswerten
%UTF8 Support auf Windows quasi nicht vorhanden, läuft nur weil LINUX Server
%ausschließliches Arbeiten mit Strings, keine Typsicherheit
%schlechte Fehlermeldungen (keine Exceptions sondern Strings), schlechte Benennungen (star) anstatt param
%nicht performant (reguläre Ausdrücke), gerade bei riesigen Bots sehr problematisch
%mehrere Usereingaben direkt auf eine Routine im .rive matchen
%matcht immer auf die erste Übereinstimmung, erschwerte Fehlersuche 
Die Umwandlung von Sprache zu Text ist bekannterweise nicht einfach und somit ist es toll, dass Google diesen Service anbietet. Die Integration war nach einer ersten Einarbeitung nicht schwierig, obwohl sich das Projekt noch im Alpha-Stadium befindet. Der Service versteht die Sprache zuverlässig.
Dadurch, dass Telegram vor allem im mobilen Umfeld genutzt wird, ist ein solche Feature besonders praktisch, da das Tippen auf mobilen Geräten nicht so genau und einfach ist, wie auf einer gewöhnlichen Tastatur. Ob der Einsatz von Google Services die Kosten sprengen würde, wird in dieser Projektarbeit nicht weiter betrachtet, da von vornherein die Implementierung als Proof of Concept gesehen wurde.

Dass Telegram unterschiedliche Dateiformate nutzt ist auf den ersten Blick nicht intuitiv, hier ist aber auch Google nicht ganz unschuldig, da die Frequenz einer Audiodatei auf Serverseite erkannt werden sollte, diese Einstellungsmöglichkeit sollte nicht dem Nutzer überlassen werden.

Insgesamt ist es aber sehr praktisch, dass man Sprachnachrichten auch an Bots senden kann, diese können somit noch umfangreichere Aufgaben übernehmen. 

\section{Ausblick}
Man kann davon ausgehen, dass nicht alle möglichen Formulierungen und Fragen der Benutzer vom Bot abgedeckt sind. Dies kann erst nach der Produktivsetzung des Bots, wenn der Zugriff mehr Leuten möglich ist, festgestellt werden. Interessant wäre also die nicht erkannten Formulierungen der Benutzer auszuwerten und so den Bot kontinuierlich zu verbessern. Die dafür notwendige Logging-Funktion ist dafür schon implementiert.

Allgemein könnten die Ausgaben des Bots an die Benutzer noch konkretisiert werden. Die Stundenplanausgabe von Studierenden, die sich im sechsten Semester im Bachelor befinden, ist eigentlich nicht sinnvoll, da dies hauptsächlich aus Wahlfächern besteht. In diese Richtung ist theoretisch noch Verbessungsspielraum. 
Es stellt sich jedoch die Frage, ob der Funktionsumfang dann nicht den Sinn übersteigt. Durch die eingeschränkten Styling- und Kommunikationsfunktionen eines Telegram-Bots kann es eher eine User-Experience-Hölle sein. Daher ist es nicht sinnvoll für einen Bot, die Möglichkeit eines selbst zusammengestellten Stundenplans zu geben.

%TODO basic authentification option?