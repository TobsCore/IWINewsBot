\chapter{Grundlagen}
Vor der Programmierung des Bots, ist es zuallererst notwendig, den Bot bei Telegram zu registrieren. Hierzu wird der \textbf{@BotFather}-Bot von Telegram genutzt, der sich um die Erstellung und Verwaltung von Bots kümmert. Es wurde entschieden, den Bot \textbf{@IWINewsBot} zu nennen. Über diesen Namen wird der Bot später von Nutzern angesprochen. Nach der erfolgreichen Erstellung des Bots konnte man zusätzlich noch ein Profilbild für den Bot festlegen und eine Art Begrüßungstext.

Schon an dieser Stelle wird definiert, welche Funktionen der Bot anbieten wird, indem die Befehle festgelegt und erklärt werden. Neben der \texttt{/start}- und \texttt{/stop}-Funktionalität, auf die in~\autoref{sec:StartStopCommands} noch genauer eingegangen wird, wird hier bereits der \texttt{/abo}-Befehl festgelegt, über den der Nutzer auswählen kann, welche Nachrichten er empfangen will.

\section{Umgang mit dem Bot}
Der Nutzer sucht den Bot innerhalb Telegrams mit der Suche\footnote{Suchwort: @IWINewsBot} oder extern über den Link \url{http://t.me/IWINewsBot}.
Der Nutzer beginnt anschließend die Unterhaltung mit dem Bot durch Klicken des \textit{Starten}-Buttons.
Für die Kommunikation mit einem Telegram-Bot ist es notwendig, dass eine Konversation mit dem Bot explizit von dem Nutzer gestartet wird. Dass ein Bot die Konversation startet ist nicht erlaubt, vermutlich um die Nutzung von Bots zur Verteilung von Spam-Nachrichten zu vermeiden.

Es war gewünscht, dass der User, nach dem Starten der Unterhaltung, als Abonnent registriert wird und von nun an alle Nachrichten, die auf dem \emph{Schwarzen Brett} gepostet werden, erhält. Sollte dies nicht gewünscht sein, kann er seine Einstellungen über \texttt{/abo} anpassen. Sollte er überhaupt keine Nachrichten mehr wünschen, kann das Abonnement über \texttt{/stop} beendet werden. Um das Einstellen der einzelnen Nachrichten zu vereinfachen, sollten \emph{Inline-Keyboards}\footnote{Mehr Informationen zu \emph{Inline-Keyboards} sind unter \url{https://core.telegram.org/bots/2-0-intro\#new-inline-keyboards} zu finden.} verwendet werden. Hierbei handelt es sich um Buttons, die anstelle einer Nachricht, also innerhalb des Textfeldes, erscheinen und den Bot somit noch interaktiver machen.

Der Bot wird, wie bereits durch einige Beispiele gezeigt, durch bestimmte Befehlswörter angesprochen, die alle mit einem Slash-Symbol beginnen. Dies hat den Grund, dass Telegram diese Befehle kennt, da diese bei der Erstellung über den @BotFather bereits angegeben wurden, sodass diese Befehle in der Anwendungsoberfläche nicht nur hervorgehoben werden, sondern es sogar ein eigenes Menü mit einer Übersicht aller Befehle gibt. Zudem bietet Telegram die Möglichkeit, dass Befehle in Nachrichten vom Bot anklickbar sein können. Dies wird beispielsweise in der Nachricht nach dem Start einer Unterhaltung genutzt, um dem Nutzer die Einstellungen leicht zugänglich zu machen. Der Bot antwortet mit:

''[...] Um Deine Einstellungen anzupassen, wähle \textcolor{blue}{/abo} aus.''

Dabei ist \texttt{/abo} als Hyperlink markiert, der, wenn er angeklickt wird, automatisch diesen Befehl ausführt. Durch diese Befehlswörter werden also nicht nur die \emph{Best-Practices} von Telegram eingehalten, sondern es wird ein echter Mehrwert geboten.

\section{Funktionen}
Neben der Abonnement-Funktion, sollte der Bot noch weitere Informationen ausgeben. Zum einen sollte eine Mensa-Funktion enthalten sein, die dem Nutzer den Essensplan der Mensa an der Hochschule ausgibt. Der Gedanke bei dieser Funktion war, dass man diese Nachrichten einfach in Telegram-Gruppen, in denen sich zum Beispiel Studierende austauschen, teilen kann.

Bei der Mensa-Funktion sollte aus Gründen der Einfachheit zuerst nur das Angebot des aktuellen Tages ausgegeben werden. Am Wochenende und an Feiertagen, also an Tagen an denen die Mensa geschlossen hat, wird dann eine entsprechende erklärende Information ausgegeben. Diese Funktion lässt sich über den Befehl \texttt{/mensa} aufrufen. Es sind sowohl die Preise für Studierende, als auch für Professoren/innen und Mitarbeiter/innen hinterlegt.
Nach Fertigstellung der Basisfunktionalität ist eine Erweiterung vorgesehen, die für die nächsten fünf Tage das Mensaangebot ausgibt und dabei direkt das Wochenende ausschließt. Bei Feiertagen und in der vorlesungsfreien Zeit wird weiterhin eine Meldung ausgegeben, wenn die Mensa geschlossen sein sollte.

Neben der Mensa-Funktion sollten zusätzlich faktultätsbezogene Informationen ausgegeben werden können, welche ebenfalls über den \texttt{/abo}-Befehl verwaltet werden können.

Insbesondere sollten Informationen über Professoren/innen integriert werden, die die Fragen beantworten, wo sich ihre Büros befinden und wann ihre Sprechstunden stattfinden. Diese Funktion ist unter \texttt{/profs} zu finden.
