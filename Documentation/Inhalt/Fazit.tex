\chapter{Bewertung und Ausblick}
\section{Bewertung}
Es bleibt festzuhalten, dass mithilfe der Telegram-API schnell ein funktionales Programm zu erstellen ist, das durch seine Plattformunabhängigkeit eine weite Reichweite bietet und somit viele Benutzer abholt. So werden zum Beispiel WindowsPhone-Benutzer, die oft bei App-Entwicklungen vernachlässigt werden, ebenfalls integeriert.
%TODO hier schreiben für welche Anwendungen es möglich und sinnvoll ist

Wünschenswert wären weitere Gestaltungsmöglichkeiten für Texte. Im Moment sind nur die Basisfunktionalitäten für Markdown und HTML vorhanden, welche zum Beispiel \emph{bold}, \emph{kursiv} und Highlighting von Links enthalten.
Für eine übersichtliche Darstellung für den User werden vor allem Aufzählungspunkte schmerzlich vermisst.

%TODO hier schreiben, dass nervig ist, dass die editmessagetext zwangsweise über den buttons erscheinen und der user nicht notified wird über neue Nachrichten

\section{Ausblick}
%TODO professoren verifizieren, damit sie direkt nachrichten an das schwarze brett posten können
%TODO alle oder die die es wollen Professoren in telegram mit username unter /profs auflisten, damit man direkt in telegram kommunizieren kann (secret chat)?
%TODO aktueller vekehrszustand zur hochschule von standort aus berechnen
%TODO integration von stundenplänen oder semesterplänen
