\chapter{Aufgabe und Ziel}
Viele Studenten setzen heutzutage auf Messenger Dienste, um miteinander in Kontakt zu bleiben und sich sowohl privat als auch über Hochschulaktivitäten zu informieren und auszutauschen. Hierbei ist \emph{WhatsApp} der wohl bekannteste Vertreter, aber auch andere Dienste, allen voran \emph{Telegram}, werden von immer mehr Menschen genutzt. Da diese Platfformen seit einiger Zeit auch die Programmierung von Bots zulässt, wurde von der Fakultät Informatik der Wunsch geäußert, diese Form der Kommunikation zu nutzen, um Studenten Neuigkeiten zu informieren.

Es gibt bereits ein \emph{Schwarzes Brett}, auf dem Neugikeiten von Dozenten und Mitarbeitern der Fakultät gepostet werden, sodass Studenten auch von zu Hause auf diese zugreifen können. Besonders bei dem Ausfall von Vorlesungen ist dies von Vorteil. Jedoch müssen hierbei die Nachrichten eigenständig gecheckt werden und eine Art \textit{Push}-Benachrichtung ist auch nicht möglich, also die Benachrichtigung.

Die Telegram-Plattform ist hier besonders für die Benachrichtung geeignet, da Studenten diesen Messenger als Handy App nutzen können und so überall über Neuigkeiten informiert werden können. Zudem gibt Desktop-Anwendungen, sodass auch über den Computer Nachrichten versandt und empfangen werden können.

Ziel dieser Projektarbeit war also die Implementierung eines Telegram-Bots, der Studenten über Nachrichten auf dem Schwarzen Brett zeitnah informiert, sodass er diese auf seinem Smartphone oder Computer vollständig und einfach lesen kann. Hierbei sollte es möglich sein, Nachrichtenarten zu filtern, sodass Master-Studenten keine Nachrichten, die an Bachelor-Studenten gerichtet sind, empfangen muss.

Neben dieser Primär-Funktionalität sollte es zudem möglich sein sich über das Essen in der Mensa zu informieren und Informationen zu Professoren abzurufen.

In dieser Ausarbeitung wird auf die Implementierung des Telegram Bots eingangen. Hierbei wird nicht nur die benutzte Technologie beschrieben, sondern der Fokus liegt auf der Implementierung von Funktionen.
