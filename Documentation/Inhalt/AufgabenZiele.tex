\chapter{Aufgabe und Ziel}
Viele Studierende setzen heutzutage auf Messenger-Dienste, um miteinander in Kontakt zu treten und sich sowohl privat als auch über Hochschulaktivitäten zu informieren und auszutauschen. Hierbei ist \emph{WhatsApp} der wohl bekannteste Vertreter, aber auch andere Dienste, allen voran \emph{Telegram}, werden von immer mehr Menschen genutzt. Da diese Plattform seit einiger Zeit auch die Programmierung von Bots zulässt, wurde von der Fakultät Informatik der Wunsch geäußert, diese Form der Kommunikation zu nutzen, um Studenten über Neuigkeiten zu informieren.

Es gibt bereits ein \emph{Schwarzes Brett}, auf dem Neugikeiten von Professoren/innen, Dozenten/innen und Mitarbeitern/innen der Fakultät gepostet werden, sodass Studierende auch von zu Hause auf diese zugreifen können. Besonders bei dem Ausfall von Vorlesungen ist dies von Vorteil.
Jedoch gibt es hierbei keine Möglichkeit, die Nachrichten automatisch zu erhalten, beispielsweise durch sogenannte \textit{Push}-Benachrichtungen auf das Smartphone, sondern die Nachrichten müssen eigenständig auf der Webseite nachgesehen werden.

Die Telegram-Plattform ist hier besonders für die Benachrichtung geeignet, da Studierende diesen Messenger als Handy-App nutzen können und so überall über Neuigkeiten informiert werden können. Zudem gibt es Desktop-Anwendungen für Linux, MacOs und Windows, sodass auch über den Computer Nachrichten versandt und empfangen werden können.

Ziel dieser Projektarbeit ist also die Implementierung eines Telegram-Bots, der Studierende über Nachrichten auf dem Schwarzen Brett zeitnah informiert, sodass er diese auf seinem Smartphone oder Computer vollständig und einfach lesen kann. Hierbei sollte es möglich sein, Nachrichtenarten zu filtern, sodass zum Beispiel Master-Studierende keine Nachrichten, die an Bachelor-Studierende gerichtet sind, empfangen muss.

Neben dieser Primär-Funktionalität sollte es zudem möglich sein sich über das Essen in der Mensa zu informieren und Informationen zu Professoren abzurufen.

In dieser Ausarbeitung wird auf die Implementierung des Telegram-Bots eingangen. Hierbei wird nicht nur die benutzte Technologie beschrieben, sondern auch Implementierung der Funktionen, auf der der Hauptfokus liegt.
