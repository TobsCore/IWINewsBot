\chapter{Konzeption}
Vor der Programmierung des Bots, musste man diesen zuallererst einmal bei Telegram registrieren, hierzu wurde der \textbf{@BotFather} Bot von Telegram genutzt, der sich um die Erstellung und Verwaltung von Bots kümmert. Es wurde entschieden, den Bot \textbf{@IWINewsBot} zu nennen. Über diesen Namen wird der Bot später von Nutzern angesprochen. Nach der erfolgreichen Erstellung des Bots konnte man zusätzlich noch ein Profilbild für den Bot festlegen und eine Art Begrüßungstext.

Schon an dieser Stelle konnte man festlegen, welche Funktionen der Bot anbieten wird, indem die Befehle festgelegt und erklärt werden. Neben der \texttt{/start} und \texttt{/stop} Funktionalität, auf die in~\autoref{sec:StartStopCommands} noch genauer eingegangen wird, wurde hier bereits der \texttt{/abo}-Befehlt festgelegt, über den ausgewählt wird, welche Nachrichten empfangen werden.

\section{Umgang mit dem Bot}
Für die Kommunikation mit einem Telegram-Bot ist es notwendig, dass ein Konversation mit dem Bot von dem Nutzer gestartet wird. Dass ein Bot die Konversation startet ist nicht erlaubt, vermutlich um die Nutzung von Bots zur Verteiltung von Spam-Nachrichten zu vermeiden. Wenn der Nutzer einen den \texttt{IWINewsBot gefunden hat}, entweder über die Telegram-Suche oder den Link \url{http://t.me/IWINewsBot}, kann er die Unterhaltung durch Klicken des \textit{Starten}-Buttons beginnen.

Es war gewünscht, dass nach dem Starten der Unterhalten der User als Abonennt registriert wird und von nun an alle Nachrichten die auf dem \emph{Schwarzen Brett} gepostet werden erhält. Sollte dies nicht gewünscht sein, kann er seine Einstellungen über \texttt{/abo} anpassen. Sollte er keine Nachrichten mehr wünschen, sollte das Abonement über \texttt{/stop} beendet werden. Um das Einstellen der einzelnen Nachrichten zu vereinfachen, sollten \emph{Inline-Keyboards}\footnote{Mehr Informationen zu \emph{Inline-Keyboards} sind bei~\cite{InlineKeyboards} zu finden.} verwendet werden. Hierbei handelt es sich um Buttons, die anstelle einer Nachricht erscheinen und den Bot somit noch interaktiver machen.

Der Bot wird, wie bereits durch einige Beispiele gezeigt, durch bestimmte Befehlswörter angesprochen, die alle mit einem Slash beginnen. Dies hat den Grund, dass Telegram diese Befehle kennt, wir hatten diese bei der Erstellung über den @BotFather bereits angegeben, sodass diese Befehle in der UI nicht nur hervorgehoben werden, sondern es ein eigenes Menü mit einer Übersicht aller Befehle gibt. Zudem bietet Telegram die Möglichkeit, dass Befehle in Nachrichten vom Bot klickbar sein können. Dies wird beispielsweise in der Nachricht nach dem Start einer Unterhaltung genutzt, um dem Nutzer die Einstellen schnell zugänglich gemacht werden. Der Bot antwortet mit:

''[...] Um Deine Einstellungen anzupassen, wähle \textcolor{blue}{/abo} aus.''

Dabei ist \emph{/abo} als Hyperlink markiert, der, wenn er angeklickt wird, automatisch diesen Befehl ausführt. Durch diese Befehlswörter werden also nicht nur die Best Practices von Telegram eingehalten, sondern es wird ein echter Mehrwert geboten.
